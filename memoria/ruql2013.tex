Instalamos la gema:

\begin{verbatim}
[~]$ gem install ruql
Fetching: ruql-0.0.4.gem (100%)
Successfully installed ruql-0.0.4
1 gem installed
\end{verbatim}

Creamos el fichero Ruby que contendr\'a las preguntas:

\begin{verbatim}
[~]$ cd tmp
[~/tmp]$ mkdir example
[~/tmp]$ vi example.rb
[~/tmp]$ cat example.rb 
\end{verbatim}

\begin{lstlisting}
quiz 'Example quiz' do
  
  fill_in :points => 2 do
    text 'The capital of California is ---'
    answer 'sacramento'
  end
  
  choice_answer :randomize => true do
    text  "What is the largest US state?"
    explanation "Not big enough." # for distractors without their own explanation
    answer 'Alaska'
    distractor 'Hawaii'
    distractor 'Texas', :explanation => "That's pretty big, but think colder."
  end
  
  select_multiple do
    text "Which are American political parties?"
    answer "Democrats"
    answer "Republicans"
    answer "Greens", :explanation => "Yes, they're a party!"
    distractor "Tories", :explanation => "They're British"
    distractor "Social Democrats"
  end
  
  select_multiple do
    text "Which are American political parties?"
    answer "Democrats"
    answer "Republicans"
    answer "Greens", :explanation => "Yes, they're a party!"
    distractor "Tories", :explanation => "They're British"
    distractor "Social Democrats"
  end
  
  truefalse 'The week has 7 days.', true
  truefalse 'The earth is flat.', false, :explanation => 'No, just looks that way'
\end{lstlisting}
\bigskip

Para generar el HTML versi\'on imprimible, ejecutamos el siguiente comando:

\begin{verbatim}
[~/tmp]$ ruql example.rb Html5 > example.html
\end{verbatim}

Si deseamos que nuestras preguntas se rendericen usando nuestro propio template, debemos especificarlo con la opci\'on -t:

\begin{verbatim}
[~/tmp]$ ruql example.rb Html5 -t template.html.erb > example.html
\end{verbatim}

Las especificaciones de c\'omo crear nuestro propio template se encuentran en el apartado de Gu\'{\i}a de usuario (v\'ease Ap\'endice \ref{Apendice2}).