{\bfseries {\Huge A}}\label{Apendice1:A}
\bigskip
\bigskip

\underline{API}: (\textit{Application Programming Interface} o Interfaz de Programaci\'on de Aplicaciones). Conjunto de funciones y procedimientos o m\'etodos que ofrece cierta librer\'{\i}a 
para ser utilizados por otro software como una capa de abstracci\'on. 
\bigskip

\underline{AJAX}: acr\'onimo de \textit{Asynchronous JavaScript And XML} (JavaScript asíncrono y XML). Es una t\'ecnica de desarrollo web para crear aplicaciones 
interactivas o RIA (\textit{Rich Internet Applications}). Estas aplicaciones se ejecutan en el cliente, es decir, en el navegador de los usuarios mientras se 
mantiene la comunicaci\'on as\'{\i}ncrona con el servidor en segundo plano. De esta forma es posible realizar cambios sobre las p\'aginas sin necesidad de 
recargarlas, mejorando la interactividad, velocidad y usabilidad en las aplicaciones.
\bigskip

\bigskip
{\bfseries {\Huge B}}\label{Apendice1:B}
\bigskip
\bigskip

\href{http://getbootstrap.com/}{\underline{Bootstrap}}: framework o conjunto de herramientas de software libre para dise\~{n}o de sitios y aplicaciones web. 
Contiene plantillas de dise\~{n}o con tipograf\'{\i}a, formularios, botones, cuadros, men\'us de navegaci\'on y otros elementos de dise\~{n}o basado en HTML 
y CSS, as\'{\i} como extensiones de JavaScript opcionales adicionales.
\bigskip

\bigskip
{\bfseries {\Huge C}}\label{Apendice1:C}
\bigskip
\bigskip

\underline{CDN}: (\textit{Content Delivery Network} o Red de Entrega de Contenidos). Red superpuesta de computadoras que contienen copias de datos, colocados en varios 
puntos de una red con el fin de maximizar el ancho de banda para el acceso a los datos de clientes por la red. Un cliente accede a una copia de la informaci\'on cerca 
del cliente, en contraposici\'on a todos los clientes que acceden al mismo servidor central, a fin de evitar embudos cerca de ese servidor.
\bigskip

\href{http://chaijs.com/}{\underline{Chai}}: librer\'{\i}a de asersiones TDD que puede incluirse con alg\'un framework de test JavaScript (como, por ejemplo, Mocha).
\bigskip

\underline{Cliente-Servidor}: modelo de aplicaci\'on distribuida en el que las tareas se reparten entre los proveedores de recursos o servicios, llamados servidores, y 
los demandantes, llamados clientes. Un cliente realiza peticiones a otro programa, el servidor, quien le da respuesta.
\bigskip

\href{http://codemirror.net/}{\underline{CodeMirror}}: editor de texto implementado en JavaScript para usar desde el navegador.
\bigskip

\underline{CVS}: (\textit{Concurrent Versioning System} o Sistema de Control de Versiones). Aplicaci\'on inform\'atica que implementa un sistema de control de versiones: 
mantiene el registro de todo el trabajo y los cambios en los ficheros (c\'odigo fuente principalmente) que forman un proyecto y permite que distintos desarrolladores 
(potencialmente situados a gran distancia) colaboren.
\bigskip

\underline{CSS3}: (\textit{Cascading Style Sheets} o Hoja de Estilos en Cascada). Lenguaje de hojas de estilo utilizado para describir el aspecto y el formato de un documento 
escrito en un lenguaje de marcas, como HTML. CSS3 es la versi\'on m\'as reciente de este lenguaje.
\bigskip

\underline{CSV}: (\textit{Comma-Separated Values}). Tipo de documento en formato abierto sencillo para representar datos en forma de tabla, en las que las columnas se separan 
por comas y las filas por saltos de l\'{\i}nea. 
\bigskip

\bigskip
{\bfseries {\Huge D}}\label{Apendice1:D}
\bigskip
\bigskip

\underline{DSL}: (\textit{Domain Specific Language} o Lenguaje de Dominio Espec\'{\i}fico). En desarrollo de software, un lenguaje de dominio espec\'{\i}fico es 
un lenguaje de programaci\'on o especificaci\'on de un lenguaje dedicado a resolver un problema en particular, representar un problema espec\'{\i}fico 
y proveer una t\'ecnica para solucionar una situaci\'on particular. 
\newpage

\bigskip
{\bfseries {\Huge E}}\label{Apendice1:E}
\bigskip
\bigskip

\underline{ERB}: (\textit{Embedded Ruby}). Sistema de plantillas que incrusta c\'odigo Ruby en un documento HTML. Est\'a inclu\'{\i}do en el n\'ucleo de Ruby.
\bigskip

\bigskip
{\bfseries {\Huge F}}\label{Apendice1:F}
\bigskip
\bigskip

\underline{Footer}: texto que se encuentra separado del texto principal del documento y que aparece al final de la p\'agina del mismo.
\bigskip

\underline{Fork}: en desarrollo de software, el fork es la acci\'on de realizar una copia exacta de un proyecto a partir de otro y contiene todo el c\'odigo fuente listo para
empezar un desarrollo independiente haciendo uso del mismo.
\bigskip

\underline{Framework}: (marco de trabajo). En el desarrollo de software, es una estructura conceptual y tecnol\'ogica de soporte definido, normalmente con artefactos o 
m\'odulos de software concretos, que puede servir de base para la organizaci\'on y desarrollo de software. T\'{\i}picamente, puede incluir soporte de programas, bibliotecas y 
un lenguaje interpretado, entre otras herramientas, para as\'{\i} ayudar a desarrollar y unir los diferentes componentes de un proyecto.
\bigskip

\bigskip
{\bfseries {\Huge G}}\label{Apendice1:G}
\bigskip
\bigskip

\underline{Gamificaci\'on}: uso de t\'ecnicas y din\'amicas propias de los juegos y el ocio en actividades no recreativas, como puede ser en actividades de aprendizaje.
\bigskip

\underline{Gema}: programa o biblioteca escrito en Ruby que proporciona una serie de m\'etodos para extender la funcionalidad del lenguaje Ruby.
\bigskip

\underline{Gemfile}: fichero de Ruby en el que se especifican todas las gemas de las que depende un proyecto o programa para que funcione correctamente.
\bigskip

\href{https://drive.google.com}{\underline{Google Drive}}: servicio de alojamiento de archivos accesible desde su p\'agina web mediante ordenadores o mediante aplicaciones m\'oviles.
\bigskip

\href{https://github.com}{\underline{GitHub}}: forja para alojar proyectos utilizando el Sistema de Control de Versiones {\bfseries Git}. 
\bigskip

\bigskip
{\bfseries {\Huge H}}\label{Apendice1:H}
\bigskip
\bigskip

\underline{Header}: texto que se encuentra separado del texto principal del documento y que aparece al comienzo de la p\'agina del mismo.
\bigskip

\href{https://www.heroku.com/}{\underline{Heroku}}: plataforma como servicio (\textit{PaaS}) de computaci\'on en la Nube que permite, entre otras cosas, alojar aplicaciones web.
\bigskip

\underline{HTML5}: (\textit{HyperText Markup Language}). Lenguaje de marcado para la elaboraci\'on de p\'aginas web. Es un est\'andar que sirve de referencia para la elaboraci\'on 
de p\'aginas web definiendo una estructura b\'asica y un c\'odigo para la definici\'on del contenido de la misma.
\bigskip

\bigskip
{\bfseries {\Huge J}}\label{Apendice1:J}
\bigskip
\bigskip

\underline{JavaScript}: lenguaje de programaci\'on interpretado. Se define como orientado a objetos, basado en prototipos, imperativo, d\'ebilmente tipado y din\'amico. Se utiliza 
principalmente en su forma del lado del cliente (\textit{client-side}), implementado como parte de un navegador web permitiendo mejoras en la interfaz de usuario y p\'aginas web din\'amicas.
\bigskip

\href{https://jquery.com/}{\underline{jQuery}}: librer\'{\i}a de JavaScript, que permite simplificar la manera de interactuar con los documentos HTML, manipular el \'arbol DOM, manejar eventos, 
desarrollar animaciones y agregar interacci\'on con la t\'ecnica AJAX a p\'aginas web.
\bigskip

\bigskip
{\bfseries {\Huge K}}\label{Apendice1:K}
\bigskip
\bigskip

\href{https://karma-runner.github.io/0.12/index.html}{\underline{Karma}}: librer\'{\i}a JavaScript que proporciona un entorno configurable y personalizable de testing que permite la integraci\'on 
continua de los mismos.
\bigskip

\bigskip
{\bfseries {\Huge L}}\label{Apendice1:L}
\bigskip
\bigskip

\underline{Lambda}: objeto en Ruby que representa un bloque, es decir, un trozo de co\'odigo Ruby asociado con la invocaci\'on de un m\'etodo. Las lambdas tambi\'en reciben el nombre
de Procs.
\bigskip

\bigskip
{\bfseries {\Huge M}}\label{Apendice1:M}
\bigskip
\bigskip

\href{http://www.mathjax.org/}{\underline{MathJax}}: librer\'{\i}a de JavaScript que permite mostrar signos y elementos matem\'aticos en los documentos HTML.
\bigskip

\underline{Metodolog\'{\i}a \'agil}: conjunto de m\'etodos de ingenier\'{\i}a del software basados en el desarrollo iterativo e incremental, donde los requisitos y soluciones 
evolucionan mediante la colaboraci\'on de grupos auto organizados y multidisciplinarios. Se caracterizan adem\'as por la minimizaci\'on de riesgos desarrollando software en
iteraciones cortas de tiempo.
\bigskip

\href{https://visionmedia.github.io/mocha/}{\underline{Mocha}}: framework de test para c\'odigo JavaScript.
\bigskip

\href{https://moodle.org/?lang=es}{\underline{Moodle}}: aplicaci\'on web de tipo Ambiente Educativo Virtual. Es un sistema de gesti\'on de cursos, de distribuci\'on libre, que 
ayuda a los profesores a crear comunidades de aprendizaje en l\'{\i}nea. Este tipo de plataformas tecnol\'ogicas tambi\'en se conoce como LCMS (\textit{Learning Content Management System}).
\bigskip

\bigskip
{\bfseries {\Huge O}}\label{Apendice1:O}
\bigskip
\bigskip

\href{http://oauth.net/2/}{\underline{OAuth 2.0}}: (\textit{Open Authorization}): protocolo abierto que permite la autorizaci\'on segura de una API de modo est\'andar y simple para aplicaciones
de escritorio, web y m\'oviles.
\bigskip

\href{http://opalrb.org/}{\underline{Opal}}: gema de Ruby que compila c\'odigo escrito en Ruby a c\'odigo JavaScript.
\bigskip

\href{http://code.edx.org/}{\underline{Open EdX}}: iniciativa de c\'odigo abierto creada con el fin de crear la plataforma de aprendizaje online de nueva generaci\'on, que proporcionar\'a
una educaci\'on de mejor calidad a los estudiantes de todo el mundo.
\bigskip

\bigskip
{\bfseries {\Huge P}}\label{Apendice1:P}
\bigskip
\bigskip

\underline{Proc}: objeto en Ruby que representa un bloque, es decir, un trozo de c\'odigo Ruby asociado con la invocaci\'on de un m\'etodo. Las procs tambi\'en reciben el nombre
de Lambdas.
\newpage

\bigskip
{\bfseries {\Huge R}}\label{Apendice1:R}
\bigskip
\bigskip

\underline{Renderer}: generador de cuestionarios en un determinado formato, como por ejemplo, HTML5.
\bigskip

\href{https://www.ruby-lang.org/es}{\underline{Ruby}}: lenguaje de programaci\'on interpretado, reflexivo y orientado a objetos, creado por el programador japon\'es Yukihiro "Matz" Matsumoto. 
Combina una sintaxis inspirada en Python y Perl con caracter\'{\i}sticas de programaci\'on orientada a objetos similares a Smalltalk. Comparte tambi\'en funcionalidad con otros lenguajes de 
programaci\'on como Lisp, Lua, Dylan y CLU.
\bigskip

\href{http://github.com/saasbook/ruql}{\underline{RuQL}}: (\textit{Ruby-based Quiz Generator and DSL}). Gema de Ruby que permite la elaboraci\'on de cuestionarios a partir de un fichero Ruby
escrito usando un DSL propio de la gema. Este DSL se ha usado como base para elaborar este Trabajo de Fin de Grado.
\bigskip

\bigskip
{\bfseries {\Huge S}}\label{Apendice1:S}
\bigskip
\bigskip

\href{http://www.sinatrarb.com/}{\underline{Sinatra}}: DSL que permite la r\'apida creaci\'on de aplicaciones web en Ruby.
\bigskip

\bigskip
{\bfseries {\Huge T}}\label{Apendice1:T}
\bigskip
\bigskip

\underline{TDD}: (\textit{Test-Driven Development} o Desarrollo Dirigido por Pruebas). Pr\'actica de programaci\'on que involucra otras dos pr\'acticas: escribir las pruebas primero (\textit{Test 
First Development}) y Refactorizaci\'on de c\'odigo (\textit{Refactoring}).
\bigskip

\underline{Template}: plantilla HTML en la cual se renderizar\'a (incrustar\'a) el contenido del cuestionario.
\bigskip

\href{https://rubygems.org/gems/thin}{\underline{Thin}}: gema de Ruby que implementa un servidor web r\'apido y ligero.
\bigskip

\bigskip
{\bfseries {\Huge W}}\label{Apendice1:W}
\bigskip
\bigskip

\underline{WeBrick}: librer\'{\i}a de Ruby que proporciona un sencillo servidor web. Est\'a inclu\'{\i}do en el n\'ucleo de Ruby. 
\bigskip

\underline{Web sem\'antica}: idea de a\~{n}adir metadatos sem\'anticos y ontol\'ogicos a la World Wide Web. Esas informaciones adicionales, que describen el contenido, el significado y la relaci\'on 
de los datos, se deben proporcionar de manera formal, para que sea posible evaluarlas autom\'aticamente por m\'aquinas de procesamiento. El objetivo es mejorar Internet ampliando la 
interoperabilidad entre los sistemas inform\'aticos usando {\bfseries agentes inteligentes}, es decir, programas en las computadoras que buscan informaci\'on sin necesidad de interacci\'on humana.
\bigskip

\underline{World Wide Web}: (WWW). Sistema de distribuci\'on de documentos de hipertexto o hipermedios interconectados y accesibles v\'{\i}a Internet. Con un navegador web, un usuario visualiza 
sitios web compuestos de p\'aginas web que pueden contener texto, im\'agenes, v\'{\i}deos u otros contenidos multimedia, y navega a trav\'es de esas p\'aginas usando hiperenlaces.
\bigskip

\bigskip
{\bfseries {\Huge X}}\label{Apendice1:X}
\bigskip
\bigskip

\href{http://xregexp.com/}{\underline{XRegExp}}: librer\'{\i}a JavaScript que proporciona y extiende la funcionalidad de las {\bfseries expresiones regulares} nativas de JavaScript.
\bigskip
