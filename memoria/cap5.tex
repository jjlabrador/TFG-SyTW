%%%%%%%%%%%%%%%%%%%%%%%%%%%%%%%%%%%%%%%%%%%%%%%%%%%%%%%%%%%%%%%%%%%%%%%%%%%%%%%
% Chapter 4: Sinatra renderer
%%%%%%%%%%%%%%%%%%%%%%%%%%%%%%%%%%%%%%%%%%%%%%%%%%%%%%%%%%%%%%%%%%%%%%%%%%%%%%%

%++++++++++++++++++++++++++++++++++++++++++++++++++++++++++++++++++++++++++++++

objetivos
desarrollo
resultados
problemas encontrados y soluciones


%++++++++++++++++++++++++++++++++++++++++++++++++++++++++++++++++++++++++++++++
\section{Creaci\'on del renderer Sinatra}
\label{:sec3}

Este otro renderer genera una aplicaci\'on Sinatra con todo lo necesario para ser desplegada en {\bfseries Heroku} o ejecutar localmente. 
\bigskip

\begin{enumerate}
  \item Esta aplicaci\'on guardar\'a todos los datos del cuestionario y de los alumnos permitidos para realizarlo en una hoja de c\'alculo de {\bfseries Google Drive}
  en la cuenta del profesor.

\item Posteriormente, servir\'a el cuestionario a los alumnos especificados (durante un periodo de tiempo establecido previamente). Una vez que los alumnos hayan
  completado el cuestionario, sus respuestas se guardar\'an en una copia del cuestionario solo visible para los profesores a trav\'es de Google Drive y su
  nota se guardar\'a adem\'as en la hoja de c\'alculo de Google Drive. De este modo, quedar\'a constancia de la realizaci\'on del mismo.
  \bigskip

\item Los alumnos podr\'an reintentar el cuestionario todas las veces que deseen mientras se encuentre activo.

\end{enumerate}

{\bfseries NOTA}: Es responsabilidad del profesor facilitar la nota a los alumnos en el momento que estime oportuno.
\bigskip

En el fichero del examen que recibe como entrada la gema, adem\'as de definir las preguntas, se debe especificar a los usuarios que har\'an
uso de la aplicaci\'on:

\begin{itemize}
  \item Por un lado, se deben indicar a los profesores que podr\'an desplegar el examen o consultar las notas de los alumnos. Se indicar\'a
  su email de Google en forma de \textit{string}. En caso de ser m\'ultiples profesores, se usar\'a una notaci\'on de \textit{array}.
  
  [foto]
  
  \item Por otra parte, se deber\'an indicar los alumnos permitidos para realizar el cuestionario. Se puede usar un \textit{Hash} con 
  la informaci\'on necesaria de ellos, o indicar el path de un fichero {\bfseries CSV} con los datos de los mismos.
  
  [foto]
  \bigskip
  
  El formato del fichero CSV debe ser del siguiente modo:
  
  [tabla]
  
\end{itemize}

Para resolver el gran problema de la autentificaci\'on de usuarios, se hace uso de OAuth. De este modo, delegamos todo el servicio a Google y evitamos, por tanto,
posibles brechas de seguridad que den lugar a suplantaciones de identidad o exposici\'on de datos sensibles de los usuarios a terceras personas.
\bigskip

Adem\'as, es necesario especificar un fichero \textit{config.yml} que contiene la ventana temporal en la cual estar\'a disponible
el cuestionario, el nombre del subdominio de Heroku que se desea usar para desplegar el cuestionario y la informaci\'on relativa a Google Drive:
\begin{itemize}
  \item Nombre de la hoja de c\'alculo donde se guardar\'an los datos de alumnos y preguntas y respuestas.
  \item Nombre de la carpeta que contendr\'a dicha hoja de c\'alculo.
  \item Path donde queremos que se cree la carpeta y la hoja de c\'alculo (si no existe alguna carpeta del path, se crear\'a tambi\'en).
  \item API keys necesarias para poder usar los servicios de Google, tanto la autenticaci\'on con OAuth como la escritura en Google Drive.
\end{itemize}

[foto de config.yml]
\bigskip

Finalmente, los ficheros que genera este renderer son:
\begin{itemize}
  \item El c\'odigo Ruby del servidor.
  \item Las vistas necesarias (incluyendo el cuestionario generado en HTML y un template ERB que se usar\'a para crear las copias de los cuestionarios
  realizados por los alumnos).
  \item Un Gemfile con las dependencias necesarias.
  \item Un Rakefile para automatizar tareas (de ejecuci\'on y despliegue de la aplicaci\'on).
  \item Una carpeta denominada \textit{config} con los datos de alumnos, profesores, las preguntas y respuestas y una copia del fichero \textit{config.yml}
  del cual se hara una lectura de los par\'ametros. De este modo, evitamos que las variables existentes en el c\'odigo contengan la informaci\'on sensible.
\end{itemize}

[foto del arbol de directorios y ficheros]


%---------------------------------------------------------------------------------
% \section{Problemas encontrados y soluciones}
% \label{sec:2}


% \subsection{Correcci\'on de preguntas de JavaScript en Ruby}
% \label{subsec:2.3}
% \bigskip
% 
% bla, bla, bla
% 
% \subsection{Timeout corto entre peticiones del navegador al servidor}
% \label{subsec:2.4}
% \bigskip
% 
% bla, bla, bla
% 
% \subsection{Problema de seguridad al evaluar c\'odigo Ruby en el servidor}
% \label{subsec:2.5}
% \bigskip
% 
% bla, bla, bla
% 
% \subsection{Lugar de almacenamiento de las respuestas de los alumnos}
% \label{subsec:2.5}
% \bigskip
% 
% La idea principal era almacenar todas las preguntas y respuestas de los alumnos en una base de datos pero, viendo la evoluci\'on que ha tenido la
% herramienta Google Drive y el aumento considerable de su uso por parte de docentes, decid\'{\i} sustituir las tradicionales y siempre mon\'otonas consultas a bases de datos por esta 
% herramienta de almacenamiento que permite visualizar y administrar f\'acilmente toda la informaci\'on.