\section{Gu\'ia de usuario: Seccion 1}
\label{Apendice2}

\begin{center}
\begin{footnotesize}

\begin{verbatim}
Texto
\end{verbatim}

\end{footnotesize}
\end{center}


% En el fichero del examen que recibe como entrada la gema, adem\'as de definir las preguntas, se debe especificar a los usuarios que har\'an
% uso de la aplicaci\'on:
%
% \begin{itemize}
%   \item Por un lado, se deben indicar a los profesores que podr\'an desplegar el examen o consultar las notas de los alumnos. Se indicar\'a
%   su email de Google en forma de \textit{string}. En caso de ser m\'ultiples profesores, se usar\'a una notaci\'on de \textit{array}.
%   
%   [foto]
%   
%   \item Por otra parte, se deber\'an indicar los alumnos permitidos para realizar el cuestionario. Se puede usar un \textit{Hash} con 
%   la informaci\'on necesaria de ellos, o indicar el path de un fichero {\bfseries CSV} con los datos de los mismos.
%   
%   [foto]
%   \bigskip
%   
%   El formato del fichero CSV debe ser del siguiente modo:
%   
%   [tabla]
%   
% \end{itemize}

% Adem\'as, es necesario especificar un fichero \textit{config.yml} que contiene la ventana temporal en la cual estar\'a disponible
% el cuestionario, el nombre del subdominio de Heroku que se desea usar para desplegar el cuestionario y la informaci\'on relativa a Google Drive:
% \begin{itemize}
%   \item Nombre de la hoja de c\'alculo donde se guardar\'an los datos de alumnos y preguntas y respuestas.
%   \item Nombre de la carpeta que contendr\'a dicha hoja de c\'alculo.
%   \item Path donde queremos que se cree la carpeta y la hoja de c\'alculo (si no existe alguna carpeta del path, se crear\'a tambi\'en).
%   \item API keys necesarias para poder usar los servicios de Google, tanto la autenticaci\'on con OAuth como la escritura en Google Drive.
% \end{itemize}
% 
% [foto de config.yml]
% \bigskip
% 
% Finalmente, los ficheros que genera este renderer son:
% \begin{itemize}
%   \item El c\'odigo Ruby del servidor.
%   \item Las vistas necesarias (incluyendo el cuestionario generado en HTML y un template ERB que se usar\'a para crear las copias de los cuestionarios
%   realizados por los alumnos).
%   \item Un Gemfile con las dependencias necesarias.
%   \item Un Rakefile para automatizar tareas (de ejecuci\'on y despliegue de la aplicaci\'on).
%   \item Una carpeta denominada \textit{config} con los datos de alumnos, profesores, las preguntas y respuestas y una copia del fichero \textit{config.yml}
%   del cual se hara una lectura de los par\'ametros. De este modo, evitamos que las variables existentes en el c\'odigo contengan la informaci\'on sensible.
% \end{itemize}