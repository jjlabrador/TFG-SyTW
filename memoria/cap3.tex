%%%%%%%%%%%%%%%%%%%%%%%%%%%%%%%%%%%%%%%%%%%%%%%%%%%%%%%%%%%%%%%%%%%%%%%%%%%%%%%
% Chapter 3: Mejoras del DSL original
%%%%%%%%%%%%%%%%%%%%%%%%%%%%%%%%%%%%%%%%%%%%%%%%%%%%%%%%%%%%%%%%%%%%%%%%%%%%%%%

%++++++++++++++++++++++++++++++++++++++++++++++++++++++++++++++++++++++++++++++

objetivos
desarrollo
resultados
problemas encontrados y soluciones

% Tras explicar en el cap\'{\i}tulo anterior la metodolog\'{\i}a empleada, los problemas encontrados
% durante la fase de desarrollo e implementaci\'on y las soluciones halladas, a continuaci\'on se
% detallar\'an todos los resultados obtenidos durante la fase de desarrollo del Trabajo de Fin de Grado: 
% por un lado se encuentran las mejoras realizadas al DSL original de la gema y por otro, el enriquecimiento
% de la misma con dos nuevos \textit{renderers}.

%++++++++++++++++++++++++++++++++++++++++++++++++++++++++++++++++++++++++++++++
\section{Mejoras del DSL original}
\label{3:sec1}

Fruto del estudio del c\'odigo y de ejecuciones sucesivas de la gema, se ha mejorado el funcionamiento
de la misma efectuando diversos cambios:

\begin{itemize}
  \item Correcci\'on de errores en el funcionamiento de la gema (enumerados en el cap\'{\i}tulo 2).
  \item Refactorizaci\'on de c\'odigo.
  \item A\~{n}adido manejo de excepciones tras errores de ejecuci\'on.
  \item A\~{n}adida la opci\'on en l\'inea de comandos \textit{--version} para comprobar la versi\'on de la gema. 
  \item A\~{n}adida la opci\'on en l\'inea de comandos \textit{--help} para ver la ayuda. 
\end{itemize}

%---------------------------------------------------------------------------------
% \section{Problemas encontrados y soluciones}
% \label{sec:2}
% 
% A continuaci\'on se detallan los problemas encontrados durante la implementaci\'on del Trabajo de Fin de Grado y las soluciones
% encontradas para los mismos.
% 
% \subsection{Entender el funcionamiento del c\'odigo de la gema}
% \label{subsec:2.1}
% \bigskip
% 
% {\normalsize {\bfseries Soluci\'on}}
% 
% Leer la documentaci\'on de la gema, generar cuestionarios de pruebas y estudiar el c\'odigo fuente.
% 
% \subsection{Corregir tests y funcionalidades de la gema}
% \label{subsec:2.2}
% \bigskip
% 
% {\normalsize {\bfseries Soluci\'on}}
% 
% Tras realizar el correspondiente \textit{fork} en GitHub para empezar a implementar mis modificaciones, ejecut\'e los tests de la 
% gema original para comprobar la ausencia de fallos. Al finalizar, algunos tests fallaron por lo que decid\'{\i} corregirlos. Del 
% mismo modo, algunas gemas de testing existentes en el Gemfile presentaban incompatibilidades con las nuevas versiones de Ruby, por
% lo que tambi\'en se corrig\'{\i}o.
% \bigskip
% 
% Del mismo modo, las siguientes funcionalidades de la gema fueron corregidas ya que no funcionaban correctamente:
% \begin{itemize}
%   \item La opci\'on que permite indicar si el orden de las respuestas en las preguntas de completar espacios en blaco importa o no.
%   \item La opci\'on de añadir comentarios opcionales a los textos de las preguntas.
%   \item La opci\'on \textit{raw} que permite incrustar el texto de las preguntas entre etiquetas \textless pre\textgreater \space HTML.
%   \item La opci\'on de explicaci\'on global para todos los \textit{distractors} no funcionaba.
% \end{itemize}
% 