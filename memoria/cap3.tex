%%%%%%%%%%%%%%%%%%%%%%%%%%%%%%%%%%%%%%%%%%%%%%%%%%%%%%%%%%%%%%%%%%%%%%%%%%%%%%%
% Chapter 3: Resultados
%%%%%%%%%%%%%%%%%%%%%%%%%%%%%%%%%%%%%%%%%%%%%%%%%%%%%%%%%%%%%%%%%%%%%%%%%%%%%%%

%++++++++++++++++++++++++++++++++++++++++++++++++++++++++++++++++++++++++++++++


Tras explicar en el cap\'{\i}tulo anterior la metodolog\'{\i}a empleada, los problemas encontrados
durante la fase de desarrollo e implementaci\'on y las soluciones halladas, a continuaci\'on se
detallar\'an todos los resultados obtenidos.

%++++++++++++++++++++++++++++++++++++++++++++++++++++++++++++++++++++++++++++++
\section{Enriquecimiento del DSL original}
\label{3:sec1}

Se han extendido las funcionalidades originales del DSL de la gema. A continuaci\'on se enumerar\'an
todas las nuevas caracter\'{\i}sticas:

\begin{itemize}
  \item 
  \item 
  \item 
  \item 
\end{itemize}

%++++++++++++++++++++++++++++++++++++++++++++++++++++++++++++++++++++++++++++++
\section{Creaci\'on del renderer HTMLForm}
\label{3:sec2}

Este renderer permite generar un documento HTML5 con un formulario en el que se encontran todas las
preguntas listas para ser completadas desde el navegador.

%++++++++++++++++++++++++++++++++++++++++++++++++++++++++++++++++++++++++++++++
\section{Tercer apartado de este capitulo}
\label{:sec3}
